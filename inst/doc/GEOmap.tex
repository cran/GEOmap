\documentclass{article}
\usepackage{amsmath} %Never write a paper without using amsmath for its many new commands
\usepackage{amssymb} %Some extra symbols
\usepackage{makeidx} %If you want to generate an index, automatically
\usepackage{graphicx} %If you want to include postscript graphics
%%%  \usepackage{mystyle}
%Create your own file, mystyle.sty where you put all your own \newcommand statements
%%%
%%%
\usepackage{amscd}

\usepackage{Sweave}


%%\VignetteIndexEntry{Examples from GEOmap}

%%% \usepackage{Sweave}

\begin{document}

%%%\renewcommand\floatpagefraction{.9}
%%%\renewcommand\topfraction{.9}
%%%\renewcommand\bottomfraction{.9}
%%%\renewcommand\textfraction{.1}
%%%\setcounter{totalnumber}{50}
%%%\setcounter{topnumber}{50}
%%%\setcounter{bottomnumber}{50}

\setkeys{Gin}{width=0.9\textwidth}



\numberwithin{equation}{section}

%%%   \SweaveOpts{prefix.string=rseis}



\author{Jonathan M. Lees\\
University of North Carolina, Chapel Hill\\
Department of Geological Sciences\\
CB \#3315, Mitchell Hall\\
Chapel Hill, NC  27599-3315\\
email: jonathan.lees@unc.edu\\
ph: (919) 962-0695
}
%%  \address{University of North Carolina, Chapel Hill}
%% \contact{Jonathan M. Lees}
%% \contactaddress{Department of Geological Sciences, CB #3315, Mitchell Hall, Chapel Hill, NC  27599-3315}
%% \contactemail{jonathan.lees@unc.edu}
%% \contactphone{(919) 962-0695}
\title{GEOmap: mapping and geology in R}
\date{November, 2007}

\maketitle

\begin{abstract}
Geomap software is aimed at geological applications in mapping.
\end{abstract}

\section{Introduction}
I developed a set of programs for making complex geological maps in R.
These program parallel, to a certain extent, the maps and the mapdata 
packages already available but they are different in significant ways and 
provide a slightly differnt set of the utilities.
Maps currently available in the mapdata package can be used
by GEOmap, but most of the data required by 
GEOmap is included in a separate package called geomapdata,
loaded independantly. 

The main differnces between maps and geomap is the lower demands
GEOmap has on requiring the maps information to be stored as independant 
strokes and topologically related polygones.  This step, while useful
and powerful for many applications, is onerous to set up for 
maps that are digitized on the fly, either from paper 
copies or from digital images on the screen.

The other difference is in the handling of projections.  GEOmap
has a few simple cartographic projections built in 
and can be expanded later by users.


\section{Projections}

There are  7 cartographic projections currently installed in 
GEOmap that can be called by the user and applied to data
either in the forward mode (Lat-Lon to x-y) or in the inverse
mode to go from the projected world back to geographic coordinates.

The set up of the projection is accomplished by 
running, for example,

\begin{Schunk}
\begin{Sinput}
> library(GEOmap)
\end{Sinput}
Spatial Point Pattern Analysis Code in S-Plus

 Version 2 - Spatial and Space-Time analysis
GEOmap is loaded\begin{Sinput}
> options(continue = " ")
> kliuLL = c(56.056, 160.64)
> PROJ = setPROJ(type = 2, LAT0 = kliuLL[1], LON0 = kliuLL[2], 
     LATS = NULL, LONS = NULL, DLAT = NULL, DLON = NULL, FN = 0)
\end{Sinput}
\end{Schunk}

This makes this location (Kliuchevskoi volcano in Kamchatka, Russia) the origin 
of a utm sperical projetion.
The structure PROJ must be passed as an argument to subsequent 
calls to GEOmap plotting routines and conversions.
The choices for projections can be seen by calling  projtype()
as in,
\begin{Schunk}
\begin{Sinput}
> projtype()
\end{Sinput}
[1] Projection Types
[1] 0 = None
[1] 1 = merc.sphr
[1] 2 = utm.sphr
[1] 3 = lambert.cc
[1] 4 = stereo.sphr
[1] 5 = utm.elps
[1] 6 = equid.cyl
[1] 99 = old crosson projection\end{Schunk}
And we can see the usage of the projection by loading and plotting a map.
First we plot the map with no projection, so the xy coordinates are Lat-Lon
and the map will be distorted.
\begin{Schunk}
\begin{Sinput}
> require(geomapdata)
> data(kammap)
> plotGEOmap(kammap, add = FALSE, asp = 1)
\end{Sinput}
\end{Schunk}
Next we show how to plot the map in projected form,
\begin{Schunk}
\begin{Sinput}
> plotGEOmapXY(kammap, PROJ = PROJ, add = FALSE)
\end{Sinput}
\end{Schunk}
Notice that by resizing the window the map retains the proper
aspect ration and the units are correct,

\section{Simple Map}

\section{Map Structure}

The internal structure of a GEOmap objection consists of
three elements which are lists themselves.
The raw XY coordinates are stored as long vectors
on the POINTS list.  These are all the geogrphic coordinates of the 
points int he map structure.  
The STROKES structure contains the meta data that allows one to 
acceess the POINTS and perform tasks and create graphical output.
The STROKES structure includes a set of vectors which have the following structure:

\section{Geologic Example}

The following illustrates some of the features available in GEOmap.
First we set up the data and then begin making the plot after manipulating the 
database.

\begin{Schunk}
\begin{Sinput}
> data(cosomap)
> data(faults)
> data(hiways)
> data(owens)
> data(cosogeol)
> proj = cosomap$PROJ
> plotGEOmapXY(cosomap, PROJ = proj, add = FALSE, ann = FALSE, 
     axes = FALSE)
> cosogeol = boundGEOmap(cosogeol)
> plotGEOmapXY(cosogeol, PROJ = proj, add = TRUE, ann = FALSE, 
     axes = FALSE)
> plotGEOmapXY(cosomap, PROJ = proj, add = TRUE, ann = FALSE, axes = FALSE)
> plotGEOmapXY(faults, PROJ = proj, add = TRUE, ann = FALSE, axes = FALSE)
\end{Sinput}
\end{Schunk}
The colors here are not very useful, so we can modify them by
assigning colors from a given palette, in this case
the palette of the program geotouch,
\begin{Schunk}
\begin{Sinput}
> XMCOL = setXMCOL()
> cosocolnumbers = 1:length(cosogeol$STROKES$col)
> newcol = XMCOL[cosocolnumbers]
> cosocolnums = cosogeol$STROKES$col
\end{Sinput}
\end{Schunk}
and lastly we must create a legend by matching the colors with the
symbols or names of hte units:
\begin{Schunk}
\begin{Sinput}
> ss = strsplit(cosogeol$STROKES$nam, split = "_")
> geo = unlist(lapply(ss, FUN = "getmem", mem = 1))
> UGEO = unique(geo)
> mgeo = match(geo, UGEO)
> cosogeol = boundGEOmap(cosogeol)
> gcol = paste(sep = ".", geo, cosogeol$STROKES$col)
> ucol = unique(gcol)
> spucol = strsplit(ucol, split = "\\.")
> N = length(spucol)
> names = unlist(lapply(spucol, FUN = "getmem", mem = 1))
> shades = unlist(lapply(spucol, FUN = "getmem", mem = 2))
> ORDN = order(names)
> plotGEOmapXY(cosomap, PROJ = proj, add = FALSE, ann = FALSE, 
     axes = FALSE)
> plotGEOmapXY(cosogeol, PROJ = proj, add = TRUE, ann = FALSE, 
     axes = FALSE)
> plotGEOmapXY(cosomap, PROJ = proj, add = TRUE, ann = FALSE, axes = FALSE)
> plotGEOmapXY(faults, PROJ = proj, add = TRUE, ann = FALSE, axes = FALSE)
> geoLEGEND(names[ORDN], shades[ORDN], 0.28, 0.14, 16, 6)
\end{Sinput}
\end{Schunk}


\section{Geologic Map Symbols}

Several standard geological symbols are available for plotting
specific  faults on plots.
These can be seen by executing the 
grided plot of many line dress ups:
\begin{Schunk}
\begin{Sinput}
> GEOsymbols()
\end{Sinput}
\end{Schunk}

\end{document}

